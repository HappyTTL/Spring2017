\chapter{Conclusion}\label{ch:6}
% \par In this thesis, we focus on studying the system performance in terms of outage probability and outage duration under correlated shadow fading, developing efficient ways to mitigate shadow fading, reduce outage and provide better Quality of Service (QoS) to users. Both physical layer and transportation layer performance are analyzed. For physical layer, downlink performance of single-cell model and multi-cell model are both investigated. Simulation results show that correlated shadow fading brings high outage probability and long outage duration. To mitigate shadow fading, cooperative communication and ultra-dense network are proposed. For transportation layer, legacy TCP are not performing well on mimic mmWave channel. A better scheme need to be developed. To develop a new proper TCP congestion control protocol for next generation networks, reduce congestion detection time is a key component. Based on this fact, we propose a fast end user congestion detection scheme which provides high precision when network topology does not change fast.
% \section{Main Contributions}
% \par The key contributions of this thesis are given as follows:
% \begin{itemize}
% \item In Chapter \ref{ch:CoopComm}, we investigate the correlated shadow fading problem in a single cell cellular network and shows that it could lead to correlated outage and long outage durations. A correlated outage field is presented. To mitigate shadow fading, relays can be deployed. The performance of three different relay deployments with different relay densities are studied. Theoretical analysis and simulations of outage performance are given to compare between different relay placement scenarios. Through these simulation, we showed that uniformly spaced relays perform better than the randomly spaced, due to the randomness of relay deployment. 
% \item In Chapter \ref{ch:ExpSingleCell}, we investigated how shadow fading at different positions in a cellular network is correlated. To model spatially correlated shadow fading we divided the entire range of shadow fading into a finite number of intervals. A Markov chain model is then constructed, where each interval becomes a state of the Markov chain model. This model can be used to analyze the  outage behavior at the application layer. We demonstrated that a well designed Markov chain model with an appropriate number of states corresponding to the standard deviation of the shadow fading is indeed a powerful tool to study system performance. For a single-cell system, this Markov chain model is able to analyze the system performance because there only exist autocorrelation in this scenario. 
% \item In Chapter \ref{ch:Multi}, we expand the work to investigate a multi-cell system performance given correlated shadow fading. Simulations are run to study the outage probability and outage duration distribution. First of all, the probability of two different BS layout: Grid Layout and Random Layout are investigated. We found that Grid Layout performs better than Random Layout. Secondly, outage probability given different BS densities and two different connecting strategies: Nearest BS and Strongest BS, are simulated. We conclude that connecting to Strongest BS will reduce the outage probability comparing with Nearest BS. Increasing BS density will not reduce outage probability when MU is connecting to the Strongest BS. However, when MU is connecting to the Nearest BS and the De-Correlation distance of correlated shadow fading is large enough, increasing BS density will reduce the outage probability. At last, we investigate system performance in terms of outage duration. The simulation results show that correlated shadow fading will result in long outage duration. Increasing BS density will efficiently reduce the percentage of long outage duration.
% \item In Chapter \ref{ch:TCP5G}, we propose a data driven machine learning congestion detection algorithm. Datasets are collected using NS2 for a dumbbell model with five different traffic scenarios. Five features are formatted from end user data. When the network condition is not changing, which means the propagation delay of each link is stable, our algorithm can detect congestion with high precision from the five features. In contrast, if the network condition changes, the algorithm fails to work. In both cases, rtt\_ratio is the most important feature to predict congestion. Other four features are less useful when doing the binary classification. The area under ROC curve is consistent with the CDF of rtt\_ratio. For stable network, a fast reacting congestion control algorithm can be designed based on our congestion detection algorithm for 5G mmWave communication network.

% \end{itemize}
% \section{Future Work}
% In the future, for next generation wireless communication network, mmWave channel will be used with high probability. Therefore, shadow fading will be a significant problem for next generation network. Algorithms to mitigate correlated shadow fading developed in this thesis can be directly applied to standard mmWave channel. To extend multi-cell model analysis, analysis of ultra dense network performance should will be a research direction. The fast end user congestion prediction algorithm can be used to design a fast reacting TCP congestion control algorithm. Real-time applications can be tested on top of the new designed transport layer protocol.